\documentclass{article}

\usepackage{jmlr2e}
\usepackage{natbib}
\usepackage{hyperref}
\usepackage{amssymb,amsmath,epsfig}
\usepackage{caption}
\usepackage[utf8]{inputenc}
\usepackage[french]{babel}
\usepackage[colorlinks,linkcolor=blue]{hyperref}
\usepackage{lipsum}
\sloppy

%%%%%%%%%%%%%%%%%%%%%%
\editor{Machine Learning Avancé (2022-2023)}

\begin{document}

\title{Reproduce the results of the article "Explaining and Harnessing Adversarial Examples"}

%% Group authors per affiliation:

\author{\name JIANG Chenao #1 \email chenao.jiang@etu.sorbonne-universite.fr \\
       \addr Master Ingénierie des Systèmes Intelligents \\
       Sorbonne Université\\
       Paris, France
       \AND
\name LIU Hanlin #2 \email hanlin.liu@etu.sorbonne-universite.fr \\
       \addr Master Systèmes Avancés et Robotiques \\
       Sorbonne Université\\
       Paris, France
       \AND
\name Chen Yuwang #3 \email yuwang.chen@etu.sorbonne-universite.fr \\
       \addr Master Ingénierie des Systèmes Intelligents \\
       Sorbonne Université\\
       Paris, France
       \AND
\name WANG Haoyu #4 \email haoyu.wang@etu.sorbonne-universite.fr \\
       \addr Master Systèmes Avancés et Robotiques \\
       Sorbonne Université\\
       Paris, France}
\maketitle
\href{https://github.com/LIUHanlin16895/MLA_projet}{Lien github: Projet Machine learning avancé} \\
% Abstract section

\begin{abstract}

 \cite{szegedy2013intriguing} argue that the primary cause of neural networks' vulnerability to adversarial perturbation is their linear nature, and uses this view to generate a simple and fast method of generating adversarial examples - FGSM. In this project, we focus on generating adversarial examples and confirming their impact on neural networks using the FGSM method, and implementing adversarial training of linear models as well as deep networks. After the adversarial training is completed, we compare the robustness of the neural network to the adversarial interference before and after training and confirm the effectiveness of the adversarial training.In addition,we will discuss the ability of different architectures of neural networks to resist interference.
 
\end{abstract}

\begin{keywords}
Adversarial examples, FGSM,  Adversarial training, GoogLeNet, Maxout,  Linear Model.
\end{keywords}


% INTRO section
\section{Introduction}
\label{sec:intro}

%\textbf{L'introduction présente le contexte général lié à la tâche, description de la tâche à traiter, détail des problèmes qui y sont associés.} \\

%\lipsum[2]

\cite{szegedy2013intriguing} argue that the vulnerability of machine learning models to interference from adversarial examples is due to the extreme non-linearity of deep neural networks. In line with this view, Szegedy et al.devise a fast method for generating adversarial examples, FGSM, and show that adversarial training can indeed greatly improve the robustness and accuracy of neural networks. In their study, they first explain the existence of adversarial examples for linear models, and explain the principle of FGSM. Afterwards, Szegedy et al. implement adversarial training on a variety of linear models and deep networks, and demonstrate the effectiveness and practicality of the training. 

The aim of this project is to re-implement the algorithms in the paper and to reproduce all the experimental results obtained. We have approached the reproduction of the paper in three main directions: firstly, the existence of adversarial examples. Second, the impact of adversarial attacks. Third, the practicality of adversarial training.

To do this, we first need to understand the adversarial example and its existence: we study its rationale and its generation method, FGSM, and generate an adversarial example based on a neural network model. Secondly, we look at the effect of this adversarial example on linear neural network models, in particular logistic regression networks (\textit{simple linear neural networks, softmax, etc.}), and implement the adversarial example training in these networks to see the effectiveness of the adversarial training. Thirdly, we will focus on adversarial training of deep networks and try to demonstrate that deep networks are more robust to adversarial examples than simple linear neural networks. In this section, we will work on generating adversarial examples and implementing adversarial training on Maxout, etc. Fourthly, we will compare the robustness of the adversarial examples and the results of the adversarial training on the above different architectures and hope to try more different approaches based on the adversarial training, such as  \textit{early stopping} and expanding the model to improve the accuracy of the model.\\

% METHODE section
\section{Presentation of the algorithm}
\label{sec:methode}
%\textbf{Une présentation synthétique de la solution ré-implémentée
%de l’article. Précisez et justifiez les éventuelles
%différences avec l'article de référence}\\

%\lipsum[2]

Researchers have identified a serious security concern with existing neural network models: an attacker can easily fool a neural network by adding specific noise to benign samples, often undetected. The attacker uses perturbations that are not perceptible to human vision/audition, which are sufficient to cause a normally trained model to output false predictions with high confidence, a phenomenon that researchers call adversarial attacks.
\\

Existing adversarial attacks can be classified as white-box, grey-box and black-box attacks based on the threat model. The difference between these three models lies in the information known to the attacker, and the FGSM approach is a white-box attack in which the threat model assumes that the attacker has complete knowledge of his target model, including the model architecture and parameters. The attacker can therefore create an adversarial sample directly on the target model by any means. The attacker can therefore create an adversarial sample directly on the target model by any means.



% DATA section
\section{Data}
\label{sec:data}
\textbf{Présentation des bases de données utilisées. Précisez et justifiez les éventuelles différences avec l’article de référence.}\\

\lipsum[3]

% EXPERIMENT section
\section{EXPERIMENT}
\label{sec:experimence}


\subsection{Description de l'expérience}
 
\textbf{Description de l'expérience réalisée, méthodologie et métriques d'évaluation.} \\

\lipsum[4]

\subsection{Résultats}
 
\textbf{Présentation des résultats expérimentaux obtenus et  comparaison par rapport à ceux de l’article de réference.} \\

\lipsum[5]

\subsection{Discussion}

\textbf{Discussion critique à partir des
résultats obtenus} \\

\lipsum[6]




% CONCLUSION section
\section{Conclusion}
\label{sec:conclusion}

\textbf{Conclusion : The paper "Explaining and Harnessing Adversarial Examples" presents a new explanation for the vulnerability of machine learning models, including neural networks, to adversarial examples. By reproducing the results, we can find that the primary cause of this vulnerability is the linear nature of these models, and the results support this explanation. The authors also propose a method for generating adversarial examples and show that using these examples for adversarial training can improve the performance of a maxout network on the MNIST dataset. Overall, the paper provides a new perspective on the phenomenon of adversarial examples and offers a potential solution for reducing their impact.} \\

%\lipsum[7]

% BIBLIO section
\section{Bibliographie}


Ian Goodfellow, David Warde-Farley, Mehdi Mirza, Aaron Courville, and Yoshua Bengio. Maxout networks. 28(3) :1319–1327, 17–19 Jun 2013.

Ian J Goodfellow, Jonathon Shlens, and Christian Szegedy. Explaining and harnessing adversarial examples. arXiv preprint arXiv :1412.6572, 2014.

Christian Szegedy, Wojciech Zaremba, Ilya Sutskever, Joan Bruna, Dumitru Erhan, Ian Goodfellow, and Rob Fergus. Intriguing properties of neural networks. arXiv preprint arXiv :1312.6199, 2013

Kui Ren, Tianhang Zheng, Zhan Qin, Xue Liu.Adversarial Attacks and Defenses in Deep Learning[J].Engineering,2020,6(3):346-360.

LIU Ximeng. Adversarial attacks and defenses in deep learning. Chinese Journal of Network and Information Security[J], 2020, 6(5): 36-53 doi:10.11959/j.issn.2096-109x.2020071



\bibliography{biblio}

\end{document}

