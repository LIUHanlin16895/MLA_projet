%\textbf{Présentation des bases de données utilisées. Précisez et justifiez les éventuelles différences avec l’article de référence.}\\

%\lipsum[3]
The dataset used for most of the tests in the article is the Mnist dataset. The MNIST dataset (Mixed National Institute of Standards and Technology database) is a large database of handwritten numbers collected and organized by the National Institute of Standards and Technology, it is a widely used dataset in machine learning. MNIST database has become a standard test in the field of machine learning. It contains 60,000 training images and 10,000 test images, witch are black and white images, normalised and centred with 28 pixels on each side.\\

Handwriting recognition is a difficult problem, and a good test for learning algorithms. Secondly, unlike ImageNet, the dataset contains only 10 categories and the image size is smaller, which is more useful for visualising the neural network results. This is the reason why we believe the article makes extensive use of the Mnist dataset.\\

In addition to this, the article has mentioned the ImageNet dataset used on GoogLeNet and the CIFAR-10 dataset used on Maxout. We have requested permission from the ImageNet team to use them but have not received a response, and CIFAR-10 has not been used. The dataset used for all our experiments up to now has been the Mnist dataset.