%\textbf{Présentation des bases de données utilisées. Précisez et justifiez les éventuelles différences avec l’article de référence.}\\

%\lipsum[3]
The dataset used for most of the tests in the article is the Mnist dataset. The MNIST dataset (Mixed National Institute of Standards and Technology database) is a large database of handwritten numbers collected and organized by the National Institute of Standards and Technology, has become a standard test in the field of machine learning. It contains 60,000 training images and 10,000 test images, witch are black and white images, normalised and centred with 28 pixels on each side.

Handwriting recognition is a difficult problem, and a good test for learning algorithms. Secondly, unlike ImageNet, the dataset contains only 10 categories and the image size is smaller, which is more useful for visualising the neural network results. This is the reason why we believe the article makes extensive use of the Mnist dataset.

The MNIST dataset has only ten classes, we tried to use a light version of ImageNet dataset for testing. Tiny ImageNet contains 200 classes, each class contains 500 training images, 50 validation images and 50 test images. However, due to the need to import local data that cannot be used with GPU, and the amount of data is too large and slow to train with GoogLeNet network, we only used convolutional neural network for simple training and testing. The main reproduction is still implemented based on the MNIST dataset.
