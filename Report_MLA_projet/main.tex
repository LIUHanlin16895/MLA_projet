\documentclass{article}

\usepackage{jmlr2e}
\usepackage{natbib}
\usepackage{hyperref}
\usepackage{amssymb,amsmath,epsfig}
\usepackage{caption}
\usepackage[utf8]{inputenc}
\usepackage[french]{babel}
\usepackage[colorlinks,linkcolor=blue]{hyperref}
\usepackage{lipsum}
\sloppy

%%%%%%%%%%%%%%%%%%%%%%
\editor{Machine Learning Avancé (2022-2023)}

\begin{document}

\title{Reproduce the results of the article "Explaining and Harnessing Adversarial Examples"}

%% Group authors per affiliation:

\author{\name JIANG Chenao #1 \email chenao.jiang@etu.sorbonne-universite.fr \\
       \addr Master Ingénierie des Systèmes Intelligents \\
       Sorbonne Université\\
       Paris, France
       \AND
\name LIU Hanlin #2 \email hanlin.liu@etu.sorbonne-universite.fr \\
       \addr Master Systèmes Avancés et Robotiques \\
       Sorbonne Université\\
       Paris, France
       \AND
\name Chen Yuwang #3 \email yuwang.chen@etu.sorbonne-universite.fr \\
       \addr Master Ingénierie des Systèmes Intelligents \\
       Sorbonne Université\\
       Paris, France
       \AND
\name WANG Haoyu #4 \email haoyu.wang@etu.sorbonne-universite.fr \\
       \addr Master Systèmes Avancés et Robotiques \\
       Sorbonne Université\\
       Paris, France}
\maketitle
\href{https://github.com/LIUHanlin16895/MLA_projet}{Lien github: Projet Machine learning avancé} \\
% Abstract section

\begin{abstract}

\textbf{Le résumé  synthétise en environ 200 mots la tâche abordées, les problèmes associés, la solution proposée, et les résultats principaux} \\

\lipsum[1]
\end{abstract}

\begin{keywords}
mot-clef 1, mot-clef 2, etc....
\end{keywords}


% INTRO section
\section{Introduction}
\label{sec:intro}

\textbf{L'introduction présente le contexte général lié à la tâche, description de la tâche à traiter, détail des problèmes qui y sont associés.} \\

\lipsum[2]

% METHODE section
\section{Presentation of the algorithm}
\label{sec:methode}
\textbf{Une présentation synthétique de la solution ré-implémentée
de l’article. Précisez et justifiez les éventuelles
différences avec l'article de référence}\\

\lipsum[2]

% DATA section
\section{Data}
\label{sec:data}
%\textbf{Présentation des bases de données utilisées. Précisez et justifiez les éventuelles différences avec l’article de référence.}\\

%\lipsum[3]
The dataset used for most of the tests in the article is the Mnist dataset. The MNIST dataset (Mixed National Institute of Standards and Technology database) is a large database of handwritten numbers collected and organized by the National Institute of Standards and Technology, it is a widely used dataset in machine learning. MNIST database has become a standard test in the field of machine learning. It contains 60,000 training images and 10,000 test images, witch are black and white images, normalised and centred with 28 pixels on each side.\\

Handwriting recognition is a difficult problem, and a good test for learning algorithms. Secondly, unlike ImageNet, the dataset contains only 10 categories and the image size is smaller, which is more useful for visualising the neural network results. This is the reason why we believe the article makes extensive use of the Mnist dataset.\\

In addition to this, the article has mentioned the ImageNet dataset used on GoogLeNet and the CIFAR-10 dataset used on Maxout. We have requested permission from the ImageNet team to use them but have not received a response, and CIFAR-10 has not been used. The dataset used for all our experiments up to now has been the Mnist dataset.

% EXPERIMENT section
\section{EXPERIMENT}
\label{sec:experimence}


\subsection{Description de l'expérience}
 
\textbf{Description de l'expérience réalisée, méthodologie et métriques d'évaluation.} \\

\lipsum[4]

\subsection{Résultats}
 
\textbf{Présentation des résultats expérimentaux obtenus et  comparaison par rapport à ceux de l’article de réference.} \\

\lipsum[5]

\subsection{Discussion}

\textbf{Discussion critique à partir des
résultats obtenus} \\

\lipsum[6]




% CONCLUSION section
\section{Conclusion}
\label{sec:conclusion}

\textbf{Conclusion : The paper "Explaining and Harnessing Adversarial Examples" presents a new explanation for the vulnerability of machine learning models, including neural networks, to adversarial examples. By reproducing the results, we can find that the primary cause of this vulnerability is the linear nature of these models, and the results support this explanation. The authors also propose a method for generating adversarial examples and show that using these examples for adversarial training can improve the performance of a maxout network on the MNIST dataset. Overall, the paper provides a new perspective on the phenomenon of adversarial examples and offers a potential solution for reducing their impact.} \\

%\lipsum[7]

% BIBLIO section
\section{Bibliographie}


Ian Goodfellow, David Warde-Farley, Mehdi Mirza, Aaron Courville, and Yoshua Bengio. Maxout networks. 28(3) :1319–1327, 17–19 Jun 2013.

Ian J Goodfellow, Jonathon Shlens, and Christian Szegedy. Explaining and harnessing adversarial examples. arXiv preprint arXiv :1412.6572, 2014.

Christian Szegedy, Wojciech Zaremba, Ilya Sutskever, Joan Bruna, Dumitru Erhan, Ian Goodfellow, and Rob Fergus. Intriguing properties of neural networks. arXiv preprint arXiv :1312.6199, 2013

Kui Ren, Tianhang Zheng, Zhan Qin, Xue Liu.Adversarial Attacks and Defenses in Deep Learning[J].Engineering,2020,6(3):346-360.

LIU Ximeng. Adversarial attacks and defenses in deep learning. Chinese Journal of Network and Information Security[J], 2020, 6(5): 36-53 doi:10.11959/j.issn.2096-109x.2020071



\bibliography{biblio}

\end{document}

